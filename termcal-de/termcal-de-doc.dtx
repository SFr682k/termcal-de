%%  This is file 'termcal-de-doc.dtx', Version 2018-03-16
%%  Copyright 2017-18 Sebastian Friedl <sfr682k@t-online.de>
%% 
%%  This work may be distributed and/or modified under the conditions of the LaTeX Project
%%  Public License, either version 1.3c of this license or (at your option) any later version.
%%  The latest version of this license is available at
%%      http://www.latex-project.org/lppl.txt
%%  and version 1.3c or later is part of all distributions of LaTeX version 2008-05-04 or later
%%
%%  This work has the LPPL maintenace status 'maintained'.
%%  The current maintainer of this work is Sebastian Friedl.
%%
%%  This work consists of the files termcal-de.dtx, termcal-de.ins,
%%  the derived file termcal-de.sty and termcal-de-doc.dtx
%%
%%  -------------------------------------------------------------------------------------------
%%
%%  The termcal-de package provides a German localization to the termcal package written by
%%  Bill Mitchell, which is intended to print a term calendar for use in planning a class.
%%
%%  -------------------------------------------------------------------------------------------
%%
%%  Please report bugs and other problems as well as suggestions for improvements
%%  to my email address (sfr682k@t-online.de).
%%
%%  -------------------------------------------------------------------------------------------


\documentclass[11pt]{ltxdoc}

\usepackage{iftex}
\RequireLuaTeX

\usepackage[no-math]{fontspec}
\usepackage{polyglossia}
\setdefaultlanguage{english}
\usepackage[english]{selnolig}

\usepackage{csquotes}
\usepackage{hologo}
\usepackage[unicode, pdfborder={0 0 0}, linktoc=all, hyperindex=false]{hyperref}

\parindent0pt

\usepackage[erewhon]{newtxmath}
\setmainfont{erewhon}
\setsansfont[Scale=MatchLowercase]{Source Sans Pro}
\setmonofont[Scale=MatchLowercase]{Hack}

\usepackage[left=4.50cm,right=2.75cm,top=3.25cm,bottom=2.75cm,nohead]{geometry}


\hyphenation{ori-ginal}


\MakeShortVerb{"}
\CheckSum{131}

\renewcommand*{\usage}[1]{\hyperpage{#1}}
\renewcommand*{\main}[1]{\hyperpage{#1}}
\newcommand*{\opt}[1]{\texttt{#1}\index{#1=\texttt{#1}|main}}

\PageIndex
\RecordChanges
\CodelineNumbered


\title{The \texttt{termcal-de} package \\ {\large\url{https://github.com/SFr682k/termcal-de}}}
\author{Sebastian Friedl \\ \href{mailto:sfr682k@t-online.de}{\ttfamily sfr682k@t-online.de}}
\date{2018/03/16 (v2.0)}

\hypersetup{pdftitle={The termcal-de package},pdfauthor={Sebastian Friedl}}

\begin{document}
    \maketitle
    \thispagestyle{empty}
    
    \begin{center} \itshape
        \enquote{To achieve great things, two things are needed; \\ a plan, and not quite enough time} \\
        --- \textsc{\upshape Leonard Bernstein} ---
    \end{center}
    
    \medskip
    \begin{abstract}
        \hspace{-1.5em}%
        The "termcal-de" package provides a German localization to the "termcal" package written by Bill Mitchell, which is intended to print a term calendar for use in planning a class.
    \end{abstract}
    
    
    \tableofcontents
    
    \clearpage
    
    
    \subsection*{Dependencies and other requirements}
    \addcontentsline{toc}{subsection}{Dependencies and other requirements}
    The "termcal-de" package requires \LaTeXe\ and the following packages:
    
    \begin{description}
        \item[\texttt{termcal}]
            The main "termcal" package
    
        \item[\texttt{pgfkeys}, \texttt{pgfopts}]
            Packages required for defining key-value sets and processing them as package options
    
        \item[\texttt{datetime2}, \texttt{datetime2-german}]
            "termcal-de" uses "datetime2" and its German language module, "datetime2-german", to print the date to the calendar cells.
            Please ensure that at least version 2.0 of "datetime2-german" is installed.
    \end{description}
    
    \subsection*{Installation}
    \addcontentsline{toc}{subsection}{Installation}
    Extract the package file first:
    \begin{enumerate}
        \item Run \LaTeX\ over the file "termcal-de.ins"
        \item Move the resulting ".sty" file to "TEXMF/tex/latex/termcal-de/"
    \end{enumerate}
    
    Then, you can compile the documentation yourself by executing \\[\smallskipamount]
    "lualatex termcal-de-doc.dtx" \\
    "makeindex -s gind.ist termcal-de-doc.idx" \\
    "makeindex -s gglo.ist -o termcal-de-doc.gls termcal-de-doc.glo" \\
    "lualatex termcal-de-doc.dtx" \\
    "lualatex termcal-de-doc.dtx"
    
    \smallskip
    or just use the precompiled documentation shipped with the source files. \\
    In both cases, copy the files "termcal-de-doc.pdf" and "README.md" to \\
    "TEXMF/doc/latex/termcal-de/"
    
    
    \subsection*{License}
    \begin{small}
        \addcontentsline{toc}{subsection}{License}
        \textcopyright\ 2017-18 Sebastian Friedl
        
        \smallskip
        This work may be distributed and/or modified under the conditions of the \LaTeX\ Project Public License, either version 1.3c of this license or (at your option) any later version.
        
        \smallskip
        The latest version of this license is available at \url{http://www.latex-project.org/lppl.txt} and version 1.3c or later is part of all distributions of \LaTeX\ version 2008-05-04 or later.
        
        \smallskip
        This work has the LPPL maintenace status \enquote*{maintained}. The current maintainer of this work is Sebastian Friedl. \\
        This work consists of the following files:
        \begin{itemize} \itemsep 0pt
            \item "termcal-de.dtx",
            \item "termcal-de.ins",
            \item the derived file "termcal-de.sty" and
            \item "termcal-de-doc.dtx"
        \end{itemize}
    \end{small}





    % DOCUMENTATION PART ----------------------------------------------------------------------
    \clearpage
    \part{The documentation}
    \section{Getting started}
    \subsection{Loading the package}
    Load "termcal-de" with "\usepackage{termcal-de}" \textit{after loading "babel" or "polyglossia"}. Now, "termcal-de" looks for "termcal" and loads it when necessary.
    
    \medskip
    "termcal-de" only adds a German localization to the \texttt{termcal} package. \\
    So, if you are already familiar with "termcal", you should read section \ref{sec:differences} about differences to plain "termcal" \emph{in any case}.
    
    \medskip
    However, if you never used "termcal", you could \dots
    \begin{itemize}
        \item[a)]
            first read \href{http://mirrors.ctan.org/macros/latex/contrib/termcal/termcal.pdf}{"termcal"'s documentation} and take a look at section \ref{sec:differences} afterwards or
        
        \item[b)]
            read the short tutorial on using "termcal" with "termcal-de" in section \ref{sec:tutorial}
    \end{itemize}
    
    
    \subsection{Package options} \label{sec:options}
    \textbf{TODO:} Provide a package option allowing to use abbreviated month names
    
    \subsubsection*{How to read this section -- an example}
    The key-value options provided by "termcal-de" are depicted as follows:
    
    \begin{itemize}
        \item "metasyntacticals" \dotfill \underline{"foo"}, \texttt{\bfseries bar}, "foobar" \\[\smallskipamount]
            Below the first line a short description of the option's effect is given.
            
            \smallskip
            \textit{But how should one interpret the first line?} \\
            That's quite simple since everything is based on this basic principle:
            \begin{enumerate}
                \item
                    The \textbf{\itshape key's name} is printed on the left hand side of the dotted line using typewriter font.
                    In this case, the key's name is "metasyntacticals" and you can change its value using "\usepackage[metasyntacticals=…]{termcal-de}".
                \item
                    \textbf{\itshape Possible values} for this key are printed on the right hand side of the dotted line.
                    In this case, valid key-value-specificatios would be "metasyntacticals=foo", "metasyntacticals=bar" and "metasyntacticals=foobar".
                \item
                    When using a \textbf{\itshape key without a value specified}, the \underline{"underlined"} value is assumed.
                    Therefore, in this example "\usepackage[metasyntacticals]{termcal-de}" is equal to "\usepackage[metasyntacticals=foo]{termcal-de}".
                \item
                    "termcal-de"'s \textbf{\itshape default configuration set} is composed out of the \texttt{\bfseries bold} printed values of all keys listed here.
            \end{enumerate}
    \end{itemize}

    
    
    \subsubsection*{Provided key-value options}
    The following key-value options are provided for allowing configuration of "termcal-de"'s behavior:
    \begin{itemize}
        \item \opt{compat}
            \dotfill \underline{"true"}, \texttt{\bfseries false} \\[\smallskipamount]
            When "compat"'s value is set to "true", "termcal-de" will retain compatibility to the original "termcal" package and avoid changing the date format required by "termcal"'s commands.
            
        \item \opt{drawdateframe}
            \dotfill \underline{"always"}, "atNewMonth", \texttt{\bfseries never} \\[\smallskipamount]
            This option allows to configure when a frame is drawn around the date. \\
            Setting "drawdateframe"'s value to "always" will draw a frame around \emph{every} date in the calendar.
            Specifying "atNewMonth" will draw a frame around the date when the month has changed since the last cell.
            Using the "never" value will draw no frame around any date.
            
        \item \opt{datetime2} \\
            This key set allows you to configure the way "datetime2" is configured for printing dates to the single cells. \\
            Configuration is done by changing the subkeys' values: \\
            "\usepackage[datetime2={local=de-DE, numeric}]{termcal-de}"
            
            \smallskip
            The following subkeys are available:
            \begin{itemize}
                \item "local"
                    \dotfill \underline{\texttt{\bfseries useregional}}, "german", "de-DE", "de-AT", "de-CH" \\[\smallskipamount]
                    Determines the language module used by "datetime2". \\
                    When "useregional" is set, the language module will be loaded according to "babel"'s or "polyglossia"'s settings. \\
                    Otherwise, the explicitly given language module will be used.
                    
                \item "numeric"
                    \dotfill \underline{\texttt{\bfseries true}}, "false"  \\[\smallskipamount]
                    Determines whether "datetime2" uses numeric date styles.
                
                \item "frompreamble"
                    \dotfill \underline{"true"}, \texttt{\bfseries false} \\[\smallskipamount]
                    When "datetime2" is loaded and configured in your preamble, you should set this key's value to "true". Otherwise, there will be clashing package options.
                    
                    \smallskip
                    When the value of this key is "true", the keys "local" and "numeric" will be ignored.
            \end{itemize}
    \end{itemize}

    
    
    
    \section{A short tutorial} \label{sec:tutorial}
    % TODO: Write a tutorial explaining the basic usage of termcal-de's commands
    \textbf{TODO:} Write a short tutorial explaining "termcal"'s basic usage
    
    \begin{description}
        \item[Printing the current date.]
            Using the "\currentdate" command inside a cell will insert the date as printed in the top left corner of the according cell.
    \end{description}
    
    \section{Differences to plain \texttt{termcal}} \label{sec:differences}
    \paragraph{Important Note:}
    \textit{This section only applies until the "compat" option (see section \ref{sec:options}) is given.} As soon as you pass it to "termcal-de", the command's syntax stays --- as in plain "termcal" itself --- "M/D/Y".
    
    \bigskip\medskip
    When using the standard configuration "termcal-de" does not only change the format of the printed dates, it also changes the date parameter's format expected by the standard "termcal" commands. \\
    More precisely, these commands are affected:
    \begin{itemize}\itemsep0pt
        \item "\begin{calendar}{<starting date>}{<nr of weeks>}"
        \item "\options{<date>}{<option list>}"
        \item "\caltext{<date>}{<text>}"
    \end{itemize}
    
    Plain "termcal" expects "<starting date>" and "<date>" to be given in the "m/d/y" format (e.~g.~"1/10/18" for January 10, 2018). Due to redefinition in "termcal-de", both arguments, "<starting date>" and "<date>" have to be given in the "D.M.YYYY" format (for January~10, 2018: "10.1.2018"). \\
    See table \ref{dateformatcomp} for some examples.
    
    \begin{table}[ht] \centering \small \renewcommand{\arraystretch}{1.25}
        \begin{tabular}{cc}
            \textbf{plain "termcal"} & \textbf{with "termcal-de" package} \\\hline
            "\begin{calendar}{1/10/18}{4}" & "\begin{calendar}{10.1.2018}{4}" \\
            "\options{12/21/12}{\noclass}" & "\options{21.12.2012}{\noclass}" \\
            "\caltext{2/7/11}{Exam}" & "\caltext{7.2.2011}{Exam}" \\
            \hspace{.45\textwidth} & \hspace{.45\textwidth}
        \end{tabular}
        \vspace{-1.5em}
        
        \caption{\small Comparison between plain \texttt{termcal} and \texttt{termcal} extended with \texttt{termcal-de}}
        \label{dateformatcomp}
    \end{table}
    
    \begin{center}
        \framebox{\framebox{\begin{minipage}{.95\textwidth}
            \textbf{\scshape\Large Attention!!} \\[\smallskipamount]
            The date format \emph{has} to be "D.M.YYYY" (or "M/D/Y" when using the "compat" option). \\
            This means that the \emph{date specifications must not contain leading zeros}.
            
            \medskip
            \textbf{Examples:}~~ Use …
            \vspace{-\smallskipamount}
            \begin{center}
                \begin{tabular}[t]{ccccccc}
                    "5.1.2016"  &    & "1/5/16"  &            & "05.01.2016" &    & "01/05/16" \\
                    "9.11.2019" & or & "11/9/19" & instead of & "09.11.2019" & or & "11/09/19" \\
                    "14.3.2018" &    & "3/14/18" &            & "14.03.2018" &    & "03/14/18"
                \end{tabular}
            \end{center}
        \end{minipage}}}
    \end{center}
    
    
    % TODO: Move this stuff to other sections
%     \section{Additional information}
%     \subsection{Using short month names}
%     By default, \texttt{termcal-de} prints long month names (e.~g.~\enquote{Januar}) when a new month starts. Passing the \texttt{shortmonth} option to the "\usepackage{termcal-de}" command switches to printing short month names (e.~g.~\enquote{Jan} for \enquote{Januar}).
%     
%     \subsection{Printing the current date}
%     You can print the current date for a cell using the "\currentdate" command inside this cell. It produces something like \enquote{9.~Februar~2000}.
    
    
    
    % Start determining the checksum from here
    \StopEventually{%
        \clearpage
        \phantomsection
        \addcontentsline{toc}{part}{Indices}%
        \PrintChanges
        \setcounter{IndexColumns}{2}
        \IndexPrologue{\section*{Index}}
        \PrintIndex}
    \clearpage
    \part{The package code}
    \DocInput{termcal-de.dtx}
    
    
    \Finale
\end{document}

